\documentclass{beamer}

\usetheme{Warsaw}
\usepackage{tikz}

\title[Solving the Graph Coloring Problem] % (optional, only for long titles)
{Problem Solving and Search in AI: Solving the Graph Coloring
Problem}
\author{Christian Wagner, Felix Winter}

\institute
{
  TU Wien  
}
\date[SS 2015] % (optional)
{Problem Solving and Search in AI: SS 2015}
\subject{Informatik}

\def\MLine#1{\par\hspace*{-\leftmargin}\parbox{\textwidth}{\[#1\]}}

\begin{document}
\section{Title}
  \frame{\titlepage}


\section{Introduction}

\begin{frame}
  \frametitle{The graphcoloring problem}


  Short explanation on the gcoloring problem
  \end{frame}


\section{Phase 1}
  \begin{frame}
    \frametitle{Overview of phase 1}

    \begin{enumerate}
      \item Algorithm description
      \item Used Methods
        \begin{itemize}
        \item{Inference techniques}
        \item{Variable and Value Selection heuristics}
        \end{itemize}
        
      \item{Experiments}
        \begin{itemize}
        \item{Used Benchmarks}
        \item{Finding out the best variant}
        \item{Best results}
        \end{itemize}

    \end{enumerate}
  \end{frame}


  \begin{frame}
    \frametitle{Algorithm description}


  \end{frame}

% slides for phase 1 insert here
  
\section{Phase 2}
  \begin{frame}
    \frametitle{Overview of phase 2}
    \begin{enumerate}

        \item Algorithm description
        \item Used Methods
          \begin{itemize}
            \item Problem formulation
            \item Neighbourhood operator
            \item Evaluation function
            \item Parameters
            \item Constructing an initial solution
          \end{itemize}
        \item Experiments
          \begin{itemize}
          \item Parameter configuration with irace
          \item Best results
          \end{itemize}


    \end{enumerate}
  \end{frame}


\begin{frame}
    \frametitle{Algorithm description}


  \end{frame}

% slides for phase 2 insert here

  \section{Conclusion}
  \begin{frame}
    \frametitle{Conclusion and Discussion}


  \end{frame}


      %%   \begin{exampleblock}{Definition}
      %%   \emph{Constraint Optimization is the process of optimizing an objective function with respect to some variables in the presence of constraints}
      %% \end{exampleblock}

      %% \begin{itemize}
      %%   \item{Variety of practical problems can be modelled and solved}
      %%   \item{Very successful in the fields of scheduling, timetabling, routing, computational biology, ...}
      %% \end{itemize}


\begin{frame}[allowframebreaks]
  \frametitle<presentation>{References}    
  \begin{thebibliography}{10}    
  \beamertemplatebookbibitems
  \bibitem{ArtModern}
    Stuart Russel, Peter Norvig.
    \newblock {\em Artificial Intelligence. A Modern Approach}.
    \newblock Pearson Education, 2010.

  \beamertemplatearticlebibitems
  \bibitem{ParallelHybridPhd}
    Ali Ahmed Altohami Hmer.
    \newblock A Parallel Hybrid Metaheuristic Approach for Timetabling
    \newblock {\em Phd Thesis in Computer Science. University of Regina. 2013}.

  \bibitem{Sudoku}
    Rhydian Lewis.
    \newblock Metaheuristics can solve sudoku puzzles.
    \newblock {\em J. Heuristics, 13(4):387–401, 2007.}.

 

 \bibitem{Luca}
    Luca Di Gaspero.
    \newblock Modeling and Solving Constrained Optimization Problems
    \newblock {\em Slides from the Lecture ``Modeling and Solving Constrained Optimization Problems'' TU Wien SS 11.}.

  \bibitem{Poole}
    David Poole and Alan Mackworth.
    \newblock Solving CSPs Using Search
    \newblock {\em \url{http://artint.info/html/ArtInt_78.html} Retrieved 20th November 2013.}.
 
  \bibitem{Bartak}
    Barták, Roman.
    \newblock Constraint programming: In pursuit of the holy grail.
    \newblock {\em Proceedings of WDS99 (invited lecture), Prague, June. 1999}


  \end{thebibliography}
\end{frame}

\begin{frame}
\frametitle{Questions}
\centerline{Any questions?}
\end{frame}


\end{document}
















